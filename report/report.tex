\title{INF564 Project report}
\documentclass[paper=a4, fontsize=11pt]{scrartcl}
\usepackage[utf8]{inputenc}
\usepackage[T1]{fontenc}
\usepackage{fourier}

\usepackage[francais]{babel}
\usepackage[protrusion=true,expansion=true]{microtype}	
\usepackage{amsmath,amsfonts,amsthm}
\usepackage{hyperref}
\usepackage{minted}
\usepackage{graphicx}

\usepackage{sectsty}
\allsectionsfont{\centering \normalfont\scshape}

\usepackage[nottoc, notlof, notlot]{tocbibind}

\usepackage{fancyhdr}
\pagestyle{fancyplain}
\fancyhead{}											% No page header
\fancyfoot[L]{}											% Empty 
\fancyfoot[C]{}											% Empty
\fancyfoot[R]{\thepage}									% Pagenumbering
\renewcommand{\headrulewidth}{0pt}			% Remove header underlines
\renewcommand{\footrulewidth}{0pt}				% Remove footer underlines
\setlength{\headheight}{13.6pt}

\numberwithin{figure}{section}			% Figurenumbering: section.fig#
\numberwithin{table}{section}				% Tablenumbering: section.tab#

\newcommand{\horrule}[1]{\rule{\linewidth}{#1}} 	% Horizontal rule

\title{	
		\usefont{OT1}{bch}{b}{n}
		\normalfont \normalsize \textsc{INF564 : Compilation} \\ [25pt]
		\horrule{0.5pt} \\[0.4cm]
		\huge Compilateur pour Mini-C en OCaml \\
		\horrule{2pt} \\[0.5cm]
}
\author{
		\normalfont 								\normalsize
        Lo\"{i}c Gelle\\[-3pt]		\normalsize
        19 mars 2017
}
\date{}


%%% Begin document
\begin{document}
\maketitle
\section{Introduction}

Ce court rapport a pour objet de présenter le travail effectué lors du projet du cours INF564 -- Compilation. La première partie explique la structure du projet et les choix techniques retenus. La deuxième partie explique le choix des environnements de développement et de tests ainsi que ses motivations. Enfin, la dernière partie décrit les principaux problèmes rencontrés pendant le développement et les tests ainsi que les solutions qui y ont été apportées. \\

Le code source du projet est disponible sur Github\footnote{Voir \url{https://github.com/loicgelle/inf564-compiler}}.

\section{Structure du projet et choix techniques}

La structure du projet suit assez simplement les étapes qui permettent de transformer le code source du programme d'entrée en code assembleur pour la machine cible.

\end{document}